\documentclass{article}

\usepackage{hyperref}

\begin{document}

\section{GRASP}

GRASP, \emph{Greedy Randomized Adaptative Search}, é um arcabouço geral pra 
solução de problemas de otimização. O emprego da meta-heurística GRASP na
resolução de um problema consiste em:

\begin{enumerate}
    \item Construir de uma solução inicial viável usando uma heurística gulosa
    \item Buscar soluções melhores em torno da vizinhança da solução contruida.
\end{enumerate}

Estes dois procedimentos são repetidos quantas vezes se julgar adequado. Os 
metódos e heurísticas utilizados na construção da solução inicial variam de 
problema pra problema, bem como a definição de vizinhança de uma solução.

\section{Fase construtiva}

A fase construtiva da minha implementação do GRASP para o \textbf{FJSP}
aloca uma operação em uma máquina de cada vez até que todas as operações
tenham sido alocadas. No caso de operações que podem ser delegadas a mais
de uma máquina, uma das máquinas é escolhida aleatoriamente; tornando
cada iteração do GRASP em uma instância de \textbf{JSP} diferente.

Quando se escolhe qual a próxima operação vai ser alocada, essa escolha 
deve ser limitada as primeiras operações na sequência tecnológica de 
cada job para evitar quebras de restrição que levariam a soluções não
viáveis. Cada operação recebe um valor de acordo uma heurística, as \emph{n}
melhores são colocadas numa lista restritiva, de onde é escolhida, aleatoriamente,
a próxima operação alocada.

A heurística utilizada para avaliar as operação é sua ordem na sequencia tecnológica
do respectivo \emph{job}, índices menores são melhor avaliados. Dessa forma,
a construção tende a alocar as operações inicias dos \emph{jobs} - aquelas das 
quais o restante das operações do \emph{job} depende - primeiro.

\section{Busca local}

Uma solução está completamente especificada quando é definida uma orientação
para todos os arcos disjuntivos do grafo disjuntivo. São consideradas
vizinhas de uma solução, todas as soluções que estão a uma inversão de
arco disjuntivo de distância.  

A busca local implementada aqui consiste em avaliar todos os vizinhos de uma 
solução e escolher o melhor, melhor aqui significa aquele que tem o menor 
caminho crítico da origem do grafo disjuntivo até o fim, o melhor vizinho 
se torna a solução corrente e uma nova busca local é realizada em torno de
seus vizinhos. Quando a busca local não encontra nenhuma solução melhor que 
a solução corrente, o processo para, e uma nova iteração do GRASP é iniciada.

O caminho crítico é o maior caminho simples da origem até o fim do grafo 
disjuntivo. Para calcular o caminho crítico, primeiro os vertices são ordenadas
topologicamente \cite{Cormem} ao mesmo tempo que se garante que o grafo não
contém ciclos \cite{tarjan}, grafos disjuntivos com ciclos representam soluções não viáveis,
além de não ser possivel calcular maior distância em grafos com ciclos. Tendo a
ordem topologica do grafo a maior distância é calculada como em \cite{longestPath}.



\bibliographystyle{plain}
\begin{thebibliography}{9}
    \bibitem{tarjan} 
    https://en.wikipedia.org/wiki/Tarjan\%27s\_strongly\_connected\_components\_algorithm
    
    \bibitem{longestPath} 
    https://en.wikipedia.org/wiki/Longest\_path\_problem
    
    \bibitem{Cormem} 
    Thomas  H. Cormem - Introduction to Algorithms, secção 22.4 Topological sort

\end{thebibliography}

\end{document}



